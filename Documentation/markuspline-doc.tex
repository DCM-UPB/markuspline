\documentclass[11pt,a4paper,twoside]{article}

\usepackage[utf8]{inputenc}
\usepackage[T1]{fontenc}
\usepackage[english]{babel}
\usepackage{graphicx}
\usepackage{latexsym,amsmath,amssymb,amsthm}
\usepackage{makeidx}
\usepackage[usenames,dvipsnames]{color}
\usepackage[unicode=true,colorlinks=true,linkcolor=RoyalBlue,citecolor=RoyalBlue]{hyperref}
\usepackage{natbib}
\usepackage{lipsum}

\title{The Markuspline Fortran Module}
\author{Francesco Calcavecchia}

\makeindex


\begin{document}
\maketitle

The \verb+markuspline+ Fortran module follows the recipe of Ref. \cite{Holzmann2005111} for representing a spline.

The module \verb+markuspline+ depends on the LAPACK library.

The following is a short user's documentation.

\section{Declaration of the module}
First of all one has to declare the utilization of the \verb+markuspline+ module, using the instruction\\\verb+USE markuspline+\\which has to be inserted just before the \verb+IMPLICIT NONE+ command in any \verb+PROGRAM+, \verb+SUBROUTINE+, \verb+FUNCTION+, or \verb+MODULE+.

\section{Spline's declaration}
In the following we will use the variable name \verb+spl+ in our examples.
A spline \verb+spl+ must be declared as
\begin{verbatim}
TYPE(MSPLINE) :: spl
\end{verbatim}

\section{Spline's initialization}
As first step, the spline should be initialized, by using the command
\begin{verbatim}
spl=new_MSPL(M= , NKNOTS= , LA= , LB= , CUTOFF= )
\end{verbatim}
where \verb+M+ and \verb+NKNOTS+ are two \verb+INTEGER+s which refer to $m$ and $N_{\text{spline}}$ in \cite{Holzmann2005111}, while \verb+LA+ and \verb+LB+ are two \verb+REAL(KIND=8)+ which represent the range on which the spline is defined (typically \verb+LA=0.d0+). Finally, \verb+CUTOFF+ is a \verb+LOGICAL+ which specifies whether the spline should smoothly goes to zero at \verb+LB+ or not.
Alternatively, one can use the subroutine
\begin{verbatim}
CALL MSPL_new(M= , NKNOTS= , LA= , LB= , SPL=spl, CUTOFF= )
\end{verbatim}
After the initialization, the spline can be used.
One has access to the following internal variables:
\begin{itemize}
\item \verb+LOGICAL :: spl%flag_init+, it indicates whether the \verb+MSPLINE+ has been initialized or not (by using \verb+new_MSPLINE+ or \verb+MSPLINE_new+);
\item \verb+INTEGER :: spl%m+;
\item \verb+INTEGER :: spl%Nknots+;
\item \verb+REAL(KIND=8) :: spl%La+;
\item \verb+REAL(KIND=8) :: spl%Lb+;
\item \begin{verbatim}REAL(KIND=8) :: spl%x(-1:spl%Nknots+1)\end{verbatim}The grid of the spline. The index \verb+0+ corresponds to \verb+spl%La+ and the index \verb+spl%Nknots+ to \verb+spl%Lb+;
\item \verb+REAL(KIND=8) :: spl%delta+\\The distance between two consecutive points in the grid;
\item \verb+REAL(KIND=8) :: spl%t(0:spl%m,0:spl%Nknots)+\\The $t$ matrix in reference \cite{Holzmann2005111} but with exchanged indexes (or, in other words, is the transposed matrix), which by default is initialized to zero by \verb+new_MSPLINE+ and \verb+MSPLINE_new+;
\item \verb+LOGICAL :: spl%cutoff+.
\end{itemize}

\section{Fit a function}
At this point one can set the \verb+t+ matrix.
The \verb+markuspline+ module provide a tool to fit any given function of the form:
\begin{verbatim}
FUNCTION f(i,x)
   REAL(KIND=8) :: f
   INTEGER, INTENT(IN) :: i
   REAL(KIND=8), INTENT(IN) :: x
END FUNCTION f
\end{verbatim}
To accomplish this, it is sufficient to invoke the following subroutine
\begin{verbatim}
CALL MSPL_fit_function(SPL=spl,F=f)
\end{verbatim}

\section{Compute the spline and its derivatives}
After that the spline has been set, its value in any $\verb!r!\in[\verb!spl%La!,\verb!spl%Lb!]$ can be computed by invoking the function
\begin{verbatim}
compute_MSPL(SPL=spl, DERIV=0, R=r)
\end{verbatim}
which returns a \verb+REAL(KIND=8)+.
The derivatives of the spline can be computed simply by setting the value of \verb+DERIV+ accordingly (\verb+DERIV=1+ for the first derivative, \verb+DERIV=2+ for the second derivative, etc.).
Alternatively, the value of the spline function and of its derivatives can be computed by means of the subroutine
\begin{verbatim}
CALL MSPL_compute(SPL=spl, DERIV= , R=r, VAL=val)
\end{verbatim}
where \verb+val+ is a \verb+REAL(KIND=8)+. If one wants to accumulate the computed value in the variable \verb+val+, one can add the optional argument \verb+RESET=.FALSE.+, e.g.
\begin{verbatim}
CALL MSPL_compute(SPL=spl,DERIV= ,R=r,VAL=val,RESET=.FALSE.)
\end{verbatim}
This is equivalent to the instruction
\begin{verbatim}
val=val+compute_MSPL(SPL=spl,DERIV=0,R=r)
\end{verbatim}

\section{Print on file}
It is possible to print on file the values of the spline or its derivatives.
This can be useful in order to visualize it.
To do so, use
\begin{verbatim}
CALL MSPL_print_on_file(SPL= , DERIV= , FILENAME= , NPOINTS= )
\end{verbatim}
where \verb+DERIV+ and \verb+NPOINTS+ are \verb+INTEGER+s, and \verb+FILENAME+ is a string containing the name of the file where the spline values should be stored.
As a result, a file with \verb+NPOINTS+ rows containing the position $\verb!r! \in [\verb!SPL%La!, \verb!SPL%Lb!]$ and the value of the spline $\verb!SPL!(\verb!r!)$.

\section{Parameter derivative of the spline}
It is also possible to compute the parameter (or variational) derivatives with respect to the $t$ matrix.
In particular, the subroutine
\begin{verbatim}
CALL MSPL_t_deriv(SPL=spl,R=r,T_DERIV=td)
\end{verbatim}
assigns to the matrix \verb+td+, defined as
\begin{verbatim}
REAL(KIND=8) :: td(0:spl%m,0:spl%Nknots)
\end{verbatim}
the derivatives' matrix
$$
\verb!td(i,j)! = \frac{\partial}{\partial t_{ij}} \verb!spl!(\verb!r!)
$$
If one wants to accumulate the derivatives on top of the pre-existing \verb+td+ matrix, it is sufficient to add the optional variable \verb+RESET=.FALSE.+ to the arguments of the subroutine, e.g
\begin{verbatim}
CALL MSPL_t_deriv(SPL=spl,R=r,T_DERIV=td,RESET=.FALSE.)
\end{verbatim}
In this way
$$
\verb!td(i,j)! := \verb!td(i,j)! + \frac{\partial}{\partial t_{ij}} \verb!spl!(\verb!r!)
$$

\subsection{Parameter derivative for particles in a simulation box}
\label{t_deriv_npart}
The \verb+markuspline+ module contains a subroutine which computes the variational derivatives for all the distances between \verb+n+ particles in a \verb+d+ dimensional space simulation box with sizes \verb+L=(\Lx,Ly, ... \)+.
If we label the particles coordinates as \verb+X+, the invocation reads
\begin{verbatim}
CALL MSPL_boxNpart_t_deriv(SPL=spl,R=X,NDIM=d,NPART=n,LBOX=L,
                                                     T_DERIV=td)
\end{verbatim}
where \verb+n+ and \verb+d+ are \verb+INTEGER+s, whereas \verb+X(1:d,1:n)+ and \verb+L(1:d)+ are of type \verb+REAL(KIND=8)+.
Also in this case it is possible to accumulate the values of the variational derivatives by adding the optional argument \verb+RESET=.FALSE.+

\section{Gradient of the parameter derivative}
\label{gradient}
Suppose that the spline is computed on the distance between two particles \verb+i+ and \verb+j+, whose coordinates are represented in $\mathbb{R}^{\verb!d!}$:
$$
\verb!r! = || \mathbf{r}_j - \mathbf{r}_i ||
$$
It is possible to compute the quantities
$$
\verb!grad(m,n,1:d)! = \nabla_{\mathbf{j}} \frac{\partial}{\partial t_{mn}} \verb!spl!(r)
$$
for all the \verb+m+ and \verb+n+ by invoking
\begin{verbatim}
CALL MSPL_grad_t_deriv(SPL=spl,R_VEC=dist,NDIM=d,
                                 GRAD_T_DERIV=grad[,RESET= ])
\end{verbatim}
where \verb+dist(0:d)+ is of type \verb+REAL(KIND=8)+ and \verb+RESET+ is the usual \verb+LOGICAL+ optional argument.
It is important to know that \verb+dist(1:d)+ contains the vector $\mathbf{r}_j - \mathbf{r}_i$, whereas \verb+dist(0)+ contains the distance (i.e. \verb+r+).

\subsection{Gradient of the parameter derivative for particles in a simulation box}
\label{gradient_npart}
As in Sec. \ref{t_deriv_npart}, also in this case it is possible to compute the gradient for a simulation box.
This is done with
\begin{verbatim}
CALL MSPL_boxNpart_grad_t_deriv(SPL=spl, R=X, NDIM=d, NPART=n,
                              LBOX=L, GRAD_T_DERIV=grad [, RESET= ])
\end{verbatim}
However, in this case the argument has different dimension:
\begin{verbatim}
grad(0:spl%m,0:spl%Nknots,1:d,1:n)
\end{verbatim}

\section{Laplacian of the parameter derivative}
\label{laplacian}
Suppose that, similarly to Sec. \ref{gradient}, we are now interested in the laplacian
$$
\verb!lapl(m,n,1:d)! = \nabla^2_{\mathbf{j}} \frac{\partial}{\partial t_{mn}} \verb!spl!(r)
$$
This can be computed with
\begin{verbatim}
CALL MSPL_lapl_t_deriv(SPL= , R_VEC= , NDIM= , 
                                 LAPL_T_DERIV= [, RESET= ])
\end{verbatim}
where \verb+LAPL_T_DERIV(0:spl%m,0:spl%Nknots)+ is of type \verb+REAL(KIND=8)+.

\section{Gradient and laplacian of the parameter derivative}
One can simultaneously obtain the results from both Sec. \ref{gradient} and \ref{laplacian} with (i.e. both the gradient and the laplacian) with the command
\begin{verbatim}
CALL MSPL_grad_and_lapl_t_deriv(SPL= , R_VEC= , NDIM= , 
                     GRAD_T_DERIV , LAPL_T_DERIV [, RESET= ])
\end{verbatim}

\subsection{Gradient and laplacian of the parameter derivative for particles in a simulation box}
As in Sec. \ref{t_deriv_npart} and \ref{gradient_npart}, we can compute all the gradients and laplacians in a simulation box with
\begin{verbatim}
CALL MSPL_boxNpart_grad_and_lapl_t_deriv(SPL= , R= , NDIM= , 
NPART= , LBOX= , GRAD_T_DERIV= , LAPL_T_DERIV= [, RESET= ])
\end{verbatim}
where one has to be aware of the dimensionality of the two following arrays:
\begin{verbatim}
GRAD_T_DERIV(0:SPL%m,0:SPL%Nknots,1:NDIM,1:NPART)
LAPL_T_DERIV0:SPL%m,0:SPL%Nknots,1:NPART)
\end{verbatim}

\section{Carbon-copy}
A spline can be transposed to another spline with different $m$ and $N_{\text{spline}}$, by doing a "carbon-copy".
This can be accomplished by invoking
\begin{verbatim}
CALL MSPL_carbon_copy(ORIGINAL_SPL=spl_in,CC_SPL=spl_out)
\end{verbatim}
where \verb+spl_in+ and \verb+spl_out+ have type \verb+TYPE(MSPL)+.

\section{Deallocate the spline's allocated memory}
When the spline is of no use any more, its allocated memory can be freed by invoking
\begin{verbatim}
CALL MSPL_deallocate(SPL=spl)
\end{verbatim}

\section{Notes on the efficiency}
If an instruction is called a lot of times, using the subroutine instead of the function will drastically increase the performance.

\section{Debugging mode}
A final note about the routine checks that are performed internally.
By default, the \verb+markuspline+ module checks the input values provided by the user, in order avoid possible mistake (for example, if the user provides a $\verb!LA! \geq \verb!LB!$).
However, these checks can be disabled with the command
\begin{verbatim}
CALL MSPL_change_debug_mode(DEBUG_MODE=.FALSE.)
\end{verbatim}
which changes the internal variable \verb+MSPL_DEBUG_MODE+ accordingly.
This might slightly enhance the performance.
The current value of \verb+MSPL_DEBUG_MODE+ can be obtained with the function
\begin{verbatim}
get_debug_mode_MSPL()
\end{verbatim}





\addcontentsline{toc}{chapter}{\bibname}
\bibliographystyle{alpha}
\bibliography{bibl.bib}

\printindex

\end{document}